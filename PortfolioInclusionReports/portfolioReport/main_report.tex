\documentclass{IEEEtran}
\usepackage{csquotes}
\usepackage{url}
\usepackage{graphicx}
\usepackage{caption}
\usepackage{xcolor,colortbl}
\usepackage{textcomp} 
\usepackage{listings}
\usepackage{enumitem}
\usepackage{float}
\usepackage{subfig}

\begin{document}
\title{Summary}
\author{Claudio Rodriguez Rodriguez}
\maketitle


\section{Introduction}
This document provides a summary of two separate reports: 
\begin{itemize}
    \item Report 1: \textit{CSE 578: Data Visualization Project}
    \item Report 2: \textit{CSE 579: Knowledge Representation and Reasoning Project}
\end{itemize}
Each report covers distinct topics, but both focus on the use of computational tools and data to address specific problems in their respective fields.

\section{Summary of Report 1: Data Visualization Project}
The first report details a project from CSE 578: Data Visualization. The project focuses on visualizing data from the 1994 Census Bureau Database to assist UVW College in creating marketing profiles. Key goals include developing insights on income, education, and demographic trends to tailor the college's offerings to various groups.

The project makes heavy use of Python-based data visualization tools like Matplotlib and Plotly, with SHAP analysis used to explain the relationships between different features in the data. A key finding was that gender and education level show significant relationships with salary outcomes, while other factors like hours worked had little impact.

Ethical considerations also played an important role, as the SHAP analysis revealed potential biases in the data, such as an overrepresentation of certain groups. The team emphasized transparency and ethical use of data, avoiding misleading conclusions in their recommendations.

\textbf{Key Contributions:}
\begin{itemize}
    \item Developed a Jupyter Notebook for exploratory data analysis.
    \item Created visualizations such as scatter plots, stacked bar charts, and box plots to illustrate trends in the data.
    \item Used SHAP analysis to examine the impact of various features on income levels.
    \item Highlighted ethical concerns around biased data and its influence on decision-making.
\end{itemize}

\section{Summary of Report 2: Knowledge Representation and Reasoning Project}
The second report comes from CSE 579: Knowledge Representation and Reasoning. It focuses on using Answer Set Programming (ASP) to solve the Automated Warehouse problem. The project implements ASP through clingo, a tool developed by the Potsdam Answer Set Solving Collection, to showcase the power of declarative problem-solving.

In this problem, multiple robots must collaborate to deliver items in an automated warehouse. The team used clingo to model the problem space and constraints, and they verified their solution using the Asprilo framework to visualize robot movements and task execution.

The main challenge was to optimize the number of steps the robots took to complete their deliveries. The team applied the ASP methodology of generating, defining, and testing possible solutions, with a strong focus on minimizing unnecessary actions. They demonstrated that with appropriate constraints, collaboration between robots could be achieved efficiently.

\textbf{Key Contributions:}
\begin{itemize}
    \item Modeled the Automated Warehouse problem using ASP and clingo.
    \item Developed constraints to ensure efficient robot collaboration.
    \item Optimized robot movements to minimize the number of steps required for delivery.
    \item Used the Asprilo framework to visualize the solution and validate results.
\end{itemize}

\section{Conclusion}
Both reports illustrate the application of advanced computational techniques to real-world problems. The Data Visualization report provides insights into using data analysis for decision-making in an academic context, while the Knowledge Representation report demonstrates how declarative programming can solve complex combinatorial problems in logistics.

\end{document}
