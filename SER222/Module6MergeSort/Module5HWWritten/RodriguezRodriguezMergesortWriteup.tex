\documentclass{article}
\usepackage[margin=1in]{geometry}
\usepackage[T1]{fontenc}
\usepackage{csquotes}
\usepackage{url}
\usepackage{graphicx}
\usepackage{caption}
\usepackage{xcolor,colortbl}
\usepackage{textcomp}
\usepackage{listings}
\usepackage{enumitem}
\usepackage{float}
\usepackage{subfig}
\usepackage{hyperref}
\usepackage{indentfirst}
\renewcommand{\familydefault}{\sfdefault}

\restylefloat{table}

\newcommand{\mc}[2]{\multicolumn{#1}{c}{#2}}
\definecolor{Gray}{gray}{0.95}
\definecolor{LightCyan}{rgb}{0.88,1,1}

\newcolumntype{a}{>{\columncolor{Gray}}l}
\newcolumntype{b}{>{\columncolor{white}}c}

\graphicspath{{./img/}}


% Listings

\lstset{numbers=none,numberblanklines=false,columns=fullflexible,basicstyle=\ttfamily,linewidth=\columnwidth,breaklines=true}

\newcommand{\CodeSymbol}[1]{\bfseries\textcolor{violet}{#1}}   % Code associated to defining styles
\newcommand{\InitColor}[1]{\bfseries\textcolor{orange}{#1}}   % Code associated to defining styles
\newcommand{\RedColor}[1]{\bfseries\textcolor{red}{#1}}   % Code associated to defining styles
\newcommand{\PairColor}[1]{\bfseries\textcolor{blue}{#1}}   % Code associated to defining styles
\newcommand{\CustomFunction}[1]{\bfseries\textcolor{magenta}{#1}}   % Code associated to defining styles

\definecolor{codegray}{gray}{0.95}
\definecolor{commentgray}{gray}{0.35}

\makeatletter

\lstset{
  tabsize = 4, %% set tab space width
  showstringspaces = false, %% prevent space marking in strings, string is defined as the text that is generally printed directly to the console
  numbers = left, %% display line numbers on the left
  commentstyle = \color{green}, %% set comment color
  keywordstyle = \color{blue}, %% set keyword color
  stringstyle = \color{red}, %% set string color
  rulecolor = \color{black}, %% set frame color to avoid being affected by text color
  basicstyle = \small \ttfamily , %% set listing font and size
  breaklines = true, %% enable line breaking
  numberstyle = \tiny,
}


\begin{document}

\title{SER 222:\\Divide and Conquer and the Merge Concept}
\author{Claudio Rodriguez Rodriguez}
\maketitle

\section{Shuffle Performance}

The idea behind the algorithm is to break apart the array into log n slices and then randomly select from each slice. We use the same approach as the mergesort algorithm but replace the array slicing with a Queue. The idea to use a queue is because it makes it easy to "take" from the slices without worrying about resizing containers.

The array is O(n log n) because we need to check every array element when merging elements (the n) over log n slices.  

The algorithm differs from Mergesort in the "Compare" function. This difference means that we could create an Interface for a Compare function of type "Less than" and another of "Random pick" and pass it on to Mergesort, and we would save the duplicated code for breaking apart the input into log n parts. But opting for the exact slice handling means we would need to compromise by using a queue on regular compare instead of in-place comparisons.

\ \\ 
\ \\ 
\ \\ 

\begin{lstlisting}[language = Java , frame = trBL , firstnumber = last , escapeinside={(*@}{@*)}]
public static <T> ListQueue<T> _shuffle(ListQueue<T> list) {
    if (list.size() <= 1) {
        return list;
    }

    ListQueue<T> left = new ListQueue<T>(); (*@\label{slices}@*)
    ListQueue<T> right = new ListQueue<T>();
    while (!list.isEmpty()) {
        left.enqueue(list.dequeue());
        if (!list.isEmpty()) {
            right.enqueue(list.dequeue());
        }
    }
    left = _shuffle(left);
    right = _shuffle(right);

    return _shuffleMerge(left, right);
}

public static <T> ListQueue<T> _shuffleMerge(ListQueue<T> a, ListQueue<T> b) {
    ListQueue<T> shuffled = new ListQueue<T>(); (*@\label{queue}@*)
    Random random = new Random();
    while (a.size() != 0 && b.size() != 0) {
        if (random.nextBoolean()) {
            shuffled.enqueue(a.dequeue()); (*@\label{easytotake}@*)
        } else {
            shuffled.enqueue(b.dequeue()); 
        }
    }

    while(!a.isEmpty())
    {
        shuffled.enqueue(a.dequeue());
    }
    while(!b.isEmpty())
    {
        shuffled.enqueue(b.dequeue());
    }
    return shuffled;
}
\end{lstlisting}

\end{document}
