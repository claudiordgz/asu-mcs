\documentclass{article}
\usepackage[margin=1in]{geometry}
\usepackage[T1]{fontenc}
\usepackage{csquotes}
\usepackage{url}
\usepackage{graphicx}
\usepackage{caption}
\usepackage{xcolor,colortbl}
\usepackage{textcomp} 
\usepackage{pythonhighlight}
\usepackage{enumitem}
\usepackage{float}
\usepackage{subfig}
\usepackage{hyperref}
\usepackage{indentfirst}
\renewcommand{\familydefault}{\sfdefault}

\newcommand{\mc}[2]{\multicolumn{#1}{c}{#2}}
\definecolor{Gray}{gray}{0.95}
\definecolor{LightCyan}{rgb}{0.88,1,1}

\newcolumntype{a}{>{\columncolor{Gray}}l}
\newcolumntype{b}{>{\columncolor{white}}c}

\graphicspath{{./img/}}


% Listings
\definecolor{pblue}{rgb}{0.13,0.13,1}
\definecolor{pgreen}{rgb}{0,0.5,0}
\definecolor{pred}{rgb}{0.9,0,0}
\definecolor{pgrey}{rgb}{0.46,0.45,0.48}

\lstset{language=Java,
  showspaces=false,
  showtabs=false,
  breaklines=true,
  showstringspaces=false,
  breakatwhitespace=true,
  commentstyle=\color{pgreen},
  keywordstyle=\color{pblue},
  stringstyle=\color{pred},
  basicstyle=\ttfamily
}


\definecolor{codegray}{gray}{0.95}
\definecolor{commentgray}{gray}{0.35}

\makeatletter

\begin{document}

\title{SER 222: ADJ Problem 2}
\author{Claudio Rodriguez Rodriguez}
\maketitle

% task_struct - from <linux/sched.h> <-- defined here

% for_each_process - from <linux/sched/signal.h> <-- defined here (MACRO)

% list_for_each - from #include <linux/list.h> <-- defined here (MACRO)
% list_entry - from #include <linux/list.h>  <-- defined here (MACRO)
% list_head - from #include <linux/list.h>

% module_param - from #include <linux/moduleparam.h> <-- defined here (MACRO)
% MODULE_PARM_DESC - from #include <linux/moduleparam.h> <-- defined here (MACRO)

% printk - from <linux/kernel.h> 
% defined in #include <linux/printk.h>
% function

% print_header
% print_row
% init_rodriguez_rodriguez_lkm_module
% exit_rodriguez_rodriguez_lkm_module

\section{Peak of interest}

\subsection{Problem}

\textit{Design an algorithm to identify and list the ``peaks of interest'' during a video}

\subsection{Analysis}

A ``peak of interest'' is a point in a video that might be interesting to a viewer. If we identify it, we would be a step closer to an algorithm for automatically building thumbnails. We assume that we will use points with $max$ interest as possible thumbnails for a video and that we can use whatever underlying data structure to solve the problem of retrieving the max interest per video. 

\subsubsection{Assumptions}

We are given the intensity data as an array for a video already available as an input to the algorithm. 

An array represents a video of length S (in seconds) with S indices. Each index stores the intensity of the content on the screen (typically indicating high levels of motion or color contrast).

The "interest" value is $0~to~100$, with $100$ being the most interesting and $0$ being the least interesting. Examples:

\begin{itemize}
  \item \verb|Input data=[0,0,0,0,0]| would represent a video that is five seconds and contains nothing of interest. 
  \begin{itemize} 
    \item \verb|Output: []|
 \end{itemize}

 \item \verb|Input data=[0,50,100,50,0]| would represent a video that is five seconds and contains one point of interest in the middle. 
  \begin{itemize} 
    \item \verb|Output: [2]|
 \end{itemize}

 \item \verb|Input data=[75,0,0,0,75]| would represent a video that is five seconds and contains two points of interest on either side. 
  \begin{itemize} 
    \item \verb|Output: [0,4]|
 \end{itemize}
\end{itemize}


\subsubsection{Metrics}

From the problem, we observe two important metrics.

\begin{itemize}
  \item M1: The algorithm returns the expected indices as results where it finds the max interest.
  \item M2: The efficiency of the algorithm. We use the number of comparisons in the algorithm as a cost metric to measure the computational time needed for a particular design.
  \item M3: Memory efficiency of the algorithm. We use the amount of memory needed to run the algorithm as a cost metric in the form of variables required. 
\end{itemize}

\subsection{Design}

The selected solution to solve the problem is shown on the Algorithm~\ref{designsolution}.

\begin{lstlisting}[style=mypython,caption={Code for solution to retrieve max intensities},label=designsolution,captionpos=b]
  def method_find_max(input_data):
    if not input_data:
      return []
    current_max = 0
    results = []
    for i, value in enumerate(input_data):  
      if value > current_max:
        current_max = value
        results = [i]
      elif value == current_max and value != 0:
        results.append(i)
    return results
\end{lstlisting}
 
\subsection{Justification}

\subsubsection{M1}

\textbf{Case 1}: The list contains null, empty, or all 0s

The algorithm will return an empty list of results, either by exiting early of the method or by checking all 0s in the list and returning an empty array.

\textbf{Case 2}: The list contains one or multiple max values

The algorithm will keep track of the max value in the list. If a new one appears, we replace the results with a new max index. If we find a duplicate max, we add the index to the current results.

\subsubsection{M2}

Because we need to check every item in the input array, the optimal solution will run in $O(n)$, and a log time solution is impossible. Therefore, the solution satisfies M2 as well.

\subsubsection{M3}

The memory used by this solution is $O(n)$ as part of the variable to hold the current max value and the results array. It is $O(n)$ because there can be the possibility that all elements in the collection have the max intensity, and in that case, our results will contain every index in the input data.

\subsection{Alternative Design}

There is an alternative proposed solution observed in the algorithm in Listing~\ref{designsolution2}. This solution fullfils M1. It uses sorting as its internal mechanism, and if we assume we are using an $O(N~log~N)$ sorting algorithm, we will observe that performance in this solution. Unfortunately, in a merge sort solution, the memory requirements will also be $O(n)$ for merge sort and the results. This argument makes the alternative solution not as optimal as the selected solution.

\begin{lstlisting}[style=mypython,caption={Alternative solution to retrieve max intensities},label=designsolution2,captionpos=b]
  def method_sort_and_get_max(input_data):
    if not input_data:
      return []
    temp = [(i,v) for i, v in enumerate(input_data)]
    temp = sorted(temp, key=lambda k: k[1], reverse=True)
    current_max = temp[0][1]
    results = []
    i = 0
    while True and current_max != 0:
      if temp[i][1] == current_max:
        results.append(temp[i][0])
        i += 1
      else:
        break
    return results
\end{lstlisting}

\section{Merge two priority queues}

\subsection{Problem}

\textit{Design an efficient algorithm to merge two priority queues.}

\subsection{Analysis}

We expect the algorithm to merge two priority queues in the following way: the algorithm will take two priority queues as input and return a single priority queue that contains the elements of both input queues. We will make the following assumptions to simplify the problem:

\subsubsection{Assumptions}

\begin{enumerate}
  \item The type of priority queue (PQ) is the maximum PQ.
  \item The specific implementation is Sedgewick's, obtained from Ruben Acu\~na's implementation in Github.\cite{ser222:racuna1}
  \item The algorithm will assume that it operates over valid Priority Queues and produces a valid merged Priority Queue.
  \item Sedgewick's implementation does not support duplicate keys.
  \item We don't need to maintain PQ1 and PQ2.
\end{enumerate}

\subsubsection{Metrics}

\begin{itemize}
  \item M1: Correctness of Algorithm. The algorithm must be able to return a merged queue with all the elements from valid PQ 1 and valid PQ 2.
  \item M2: The efficiency of the algorithm. We use the number of comparisons in the algorithm as a cost metric to measure the computational time needed for a particular design.
\end{itemize}

\subsection{Design}

The solution to the problem is on Algorithm~\ref{designsolution3}. It assumes a proper implementation of Sedgewick's algorithm and uses $Integer$ to demonstrate the behavior. It is important to note that the Keys would not be Integer in a real-world scenario but an Object that implements the Comparable interface. 
The proper implementation is a Priority Queue as defined according to  Definition~5 in \textbf{K} that contains \verb|insert| and \verb|delMax| which work in $O(log~n)$ time. This complexity is because the underlying implementation uses a Heap-ordered Array, which is in Definition~4 of \textbf{K}. 

\begin{lstlisting}[language=Java,caption={Code for solution to retrieve max intensities},label=designsolution3,captionpos=b,frame=single]
// Step 1: Create new queue
MaxPQ<Integer> merged = new MaxPQ<Integer>(pq1.size() + pq2.size());
while (!pq1.isEmpty() && !pq2.isEmpty()) {
  Integer v1 = pq1.max();
  Integer v2 = pq2.max();
  // insert largest one first into new PQ
  if (v1 > v2) {
    merged.insert(v1);
    pq1.delMax();
  } else if (v2 > v1) {
    merged.insert(v2);
    pq2.delMax();
  } else {
    throw new UnsupportedOperationException("Queues have duplicate Priority");
  }
}

// Step 2: Insert remaining elements from PQ1 and PQ2
while (!pq1.isEmpty()) {
  merged.insert(pq1.max());
  pq1.delMax();
}

while (!pq2.isEmpty()) {
  merged.insert(pq2.max());
  pq2.delMax();
}
\end{lstlisting}

\subsection{Justification}

\subsubsection{M1}

\textbf{Case 1}: PQ1 and PQ2 contain duplicate keys

The algorithm will raise an UnsupportedOperationException to prevent the user from continuing.

\textbf{Case 2}: PQ1 and PQ2 are two valid Priority Queues.

The algorithm will insert the max value from each PQ into the merged PQ in $O(n)$ time, where N is the combined length of PQ1 and PQ2.

\subsubsection{M2}

The algorithm's efficiency is $O(n~log~n)$. The complexity is due to the initial loop where we process all elements from each Queue, and then we have a $log~n$ step inside the circle where we call $delMax$ and $insert$. 

\subsection{Alternative Design}

Unfortunately, there are not that many choices for alternative designs. Moreover, the algorithm's Complexity is $O(n~log~n)$ because of the loop to extract the elements from PQ1 and PQ2 even if we were to find a faster way to insert them into the new Queue by returning an Array of sorted elements. We could save the $insert$ call by replacing the sorted array, but the Big-Oh complexity would remain the same.

\bibliographystyle{IEEEtran}
\bibliography{IEEEabrv,RodriguezRodriguezADJ2}

\end{document}
